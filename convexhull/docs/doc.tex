% K. Kafara
\documentclass[11pt]{article}
\usepackage[margin=1.35in]{geometry}
\usepackage{polski}
\usepackage[utf8]{inputenc}
\usepackage{amsmath}
\usepackage{amsthm}
\usepackage{graphicx}
\usepackage{pgfplots}
\usepackage{multirow}
\usepackage{pdfpages}
\usepackage{listings}
\usepackage[polish]{babel}
\usepackage[normalem]{ulem}
\usepackage[backend=biber,style=alphabetic,sorting=ynt]{biblatex}
\addbibresource{bibliography.bib}


\useunder{\uline}{\ul}{}
\pgfplotsset{compat=1.9}
\theoremstyle{remark} \newtheorem{definition}{def.}
\theoremstyle{definition} \newtheorem{twierdzenie}{tw.}
\newcommand{\bold}[1]{\textbf{#1}}
\newcommand{\bemph}[1]{\textbf{\emph{#1}}}
\newcommand{\eq}{\, = \,}
\newcommand{\apeq}{\, \approx \,}
\newcommand{\miunit}{\, \frac{v \cdot s}{a \cdot m}}
\newcommand{\abs}[1]{\left| #1 \right|}
\newcommand{\bunit}{\, \mu T}
\newcommand{\iunit}{\, mA}


\usepackage{xcolor}

\definecolor{codegreen}{rgb}{0,0.6,0}
\definecolor{codegray}{rgb}{0.5,0.5,0.5}
\definecolor{codepurple}{rgb}{0.58,0,0.82}
\definecolor{backcolour}{rgb}{0.95,0.95,0.92}

\lstdefinestyle{mystyle}{
    backgroundcolor=\color{backcolour},   
    commentstyle=\color{codegreen},
    keywordstyle=\color{magenta},
    numberstyle=\tiny\color{codegray},
    stringstyle=\color{codepurple},
    basicstyle=\ttfamily\footnotesize,
    breakatwhitespace=false,         
    breaklines=true,                 
    captionpos=b,                    
    keepspaces=true,                 
    numbers=left,                    
    numbersep=5pt,                  
    showspaces=false,                
    showstringspaces=false,
    showtabs=false,                  
    tabsize=2
}

\lstset{style=mystyle}

\author{K. Kafara\\Ł. Czarniecki}
\title{\textbf{Otoczka wypukła dla zbioru punktów w przestrzeni dwuwymiarowej}\\Dokumentacja projektu\\Algorytmy geometryczne}
\date{}

\begin{document}

\maketitle



\tableofcontents

\listoffigures

\listoftables

\newpage



\section{Informacje techniczne}

\subsection{Budowa programu}

Program złożony jest z następujących modułów: 

\begin{itemize}
    \item   \emph{lib} -- biblioteczny -- zawiera zbiór pomocniczych funkcji i struktur danych wykorzystywanych przez algorytmy.
    \item   \emph{pure} -- algorytmy w \emph{czystej postaci} tj. nie posiadające części wizualizacyjnej. 
    \item   \emph{vis} -- algorytmy wraz z kodem odpowiadającym za wizualizację
\end{itemize}


Poniżej przedstawiamy dokładny opis zawartości poszczególnych modułów. 

\subsubsection{Moduł \emph{lib}}

Moduł zawiera w sobie następujące podmoduły:

\begin{enumerate}
    \item   \emph{geometric\_tool\_lab.py} -- narzędzie graficzne dostarczone w ramach przedmiotu \emph{Algorytmy geometryczne}
    \item   \emph{getrand.py} -- zawiera funkcje generujące zbiory punktów różnych typów 
    \item   \emph{sorting.py} -- zawiera implementację iteracyjnej wersji algorytmu \emph{QuickSort} wykorzystywaną m.in w algorytmie Grahama
    \item   \emph{stack.py} -- zawiera klasę implementującą \emph{stos}
    \item   \emph{util.py} -- zawiera szereg funkcji pomocniczych wykorzystywanych przez zaimplementowane algorytmy
    \item   \emph{mytypes.py} -- zawiera definicje typów stworzone w celu zwiększenia czytelności kodu
\end{enumerate}


\subsubsection{Moduł \emph{pure}}

Moduł zawiera w sobie następujące podmoduły:

\begin{enumerate}
    \item   \emph{divide\_conq.py} -- implementacja algorytmu dziel i zwyciężaj
    \item   \emph{graham.py} -- implementacja algorytmu Grahama
    \item   \emph{increase.py} -- implementacja algorytmu przyrostowego
    \item   \emph{jarvis.py} -- implementacja algorytmu Jarvisa
    \item   \emph{lowerupper.py} -- implementacja algorytmu "górna-dolna"
\end{enumerate}

\subsubsection{Moduł \emph{vis}}

Moduł zawiera w sobie następujące podmoduły:

\begin{enumerate}
    \item   \emph{divide\_conq\_vis.py} -- implementacja algorytmu dziel i zwyciężaj wraz z kodem tworzącym wizualizację
    \item   \emph{graham\_vis.py} -- implementacja algorytmu Grahama wraz z kodem tworzącym wizualizację
    \item   \emph{increase\_vis.py} -- implementacja algorytmu przyrostowego wraz z kodem tworzącym wizualizację
    \item   \emph{jarvis\_vis.py} -- implementacja algorytmu Jarvisa wraz z kodem tworzącym wizualizację
    \item   \emph{lowerupper\_vis.py} -- implementacja algorytmu "górna-dolna" wraz z kodem tworzącym wizualizację
\end{enumerate}


\subsection{Wymagania techniczne}

\begin{enumerate}
    \item   Python 3.9.0 64-bit lub nowszy
    \item   Jupyter Notebook
\end{enumerate}

\subsection{Korzystanie z programu}

\subsubsection{Uruchomienie wizualizacji}

W celu uruchomienia wizualizacji algorytmów należy uruchomić notebook (poprzez Jupyter Notebook) \emph{program.ipynb},
a następnie zapoznać się z zamieszczoną tam instrukcją. 


\section{Oznaczenia i definicje}

Na potrzeby dalszych wywodów przyjmujemy w tym miejscu szereg oznaczeń i definicji:




\section{Problem}

Wyznaczyć otoczkę wypukłą podanego zbioru punktów płaszczyzny dwuwymiarowej. 

\section{Algorytmy}

\subsection{Algorytm Grahama}

    W celu opisania sposobu działania algorytmu Grahama, definiujemy następujacą relację $\preceq_Q$ określoną dla dowolnych dwóch punktów płaszczyzny $P_1$, $P_2$ względem 
    wybranego i ustalonego punktu odniesienia $Q$.

    \begin{equation}
        \label{eq:relacja-graham}
        P_1 \preceq_Q P_2 \, \Leftrightarrow \, (\angle (P_1, Q, OX) < \angle (P_2, Q, OX)) \lor (\angle (P_1, Q, OX) \eq \angle (P_2, Q, OX) \land d(P_1, Q) <= d(P_2, Q))
    \end{equation}

    gdzie $d(P, Q)$ oznacza odległość od siebie dwóch dowolnych punktów płaszczyzny.

    Tak zdefiniowana relacja jest liniowym porządkiem (zwrotna, antysymetryczna, przechodnia i spójna).

    \subsubsection{Opis działania}

    \begin{enumerate}
        \item   Wyznaczamy najniższy punkt $Q$ wyjściowego zbioru (jeżeli jest wiele o tej samej rzędnej -- bierzemy ten o najmniejszej odciętej).
        \item   Ustawiamy go jako pierwszy element zbioru. 
        \item   Sortujemy pozostałe punkty względem relacji $\preceq_Q$.
        \item   Usuwamy wszystkie, poza najbardziej oddalonym od Q, punkty leżące na półprostej $QP$, dla każdego $P$
        \item   Kładziemy pierwsze 3 punkty zbioru na stos $S$. 
        \item   Iterujemy kolejno po punktach z posortowanego zbioru nie będących na stosie:
                Niech bieżącym punktem będzie P:

                \begin{enumerate}
                    \item   Dopóki $P$ nie jest po lewej stronie $S_{n-1}S_n$ wykonujemy (b)
                    \item   Uswamy punkt ze stosu. 
                    \item   Dodajemy $P$ na stos.
                \end{enumerate}
        \item Zwracamy zawartość stosu.
    \end{enumerate}


    \subsubsection{Szczegóły}

    \begin{itemize}
        \item   Najniższy punkt wyjściowego zbioru (punkt 1) wyznaczamy w czasie liniowym, iterując po kolejnych punktach zbioru. 
        \item   Wszystkie punkty leżacej na jednej prostej, poza najbardziej oddalonym od $Q$ usuwamy w czasie liniowym w następujący sposób:
                Iterując przez posortowaną tablicę, zaczynająć od indeksu $i := 1$, zapamiętujemy ostatni indeks na który wstawialiśmy $j$ (na początku $j := 1$).
                Jeżeli $Q$, $P_i$, $P_{i+1}$ są współliniowe to $i := i+1$. Jeżeli nie są współliniowe to $P_i$ wpisujemy na pozycję $j$, a następnie $j := j + 1$. Następnie, 
                w dalszej części algorytmu posługujemy się częścią tablicy $[0, \ldots, j - 1]$.
    \end{itemize}    


    \subsubsection{Złożoność}
    
    Operacją dominującą w algorytmie jest sortowanie -- realizowane w czasie $O(n \, lgn)$. Wybór punktu najniższego, redukcja punktów współlinowych oraz iterowanie (punkt 6, 
    zauważmy, że każdy punkt zbioru wyjściowego jest obsługiwany co najwyżej 2 razy -- gdy jest dodawany do otoczki i gdy jest ewentualnie usuwany) są realizowane w
    czasie $O(n)$. Algorytm Grahama ma zatem złożoność $O(n \, lgn)$.


    \subsubsection{Kod}


    \begin{lstlisting}[language=Python]
def get_point_cmp(ref_point: Point, eps: float = 1e-7) -> Callable:
    def point_cmp(point1, point2):
        orient = orientation(ref_point, point1, point2, eps)
        
        if orient == -1:
            return False
        elif orient == 1:
            return True
        elif dist_sq(ref_point, point1) <= dist_sq(ref_point, point2):
            return True
        else:
            return False

    return point_cmp


def graham(points: ListOfPoints) -> ListOfPoints:
    istart = index_of_min(points, 1)

    points[istart], points[0] = points[0], points[istart]    

    qsort_iterative(points, get_point_cmp(points[0]))

    i, new_size = 1, 1
    while i < len(points):
        while (i < len(points) - 1) \
        and \
        (orientation(points[0], points[i], points[i + 1], 1e-7) == 0):  
            i += 1
        
        points[new_size] = points[i]
        new_size += 1
        i += 1
    
    s = Stack()
    s.push(points[0])
    s.push(points[1])
    s.push(points[2])
    
    for i in range(3, new_size, 1):
        while orientation(s.sec(), s.top(), points[i], 1e-7) != 1:
            s.pop()

        s.push(points[i])

    return s.s[:s.itop+1]
    \end{lstlisting}



\subsection{Algorytm Jarvisa}
    \subsubsection{Opis działania}
    \begin{enumerate}
        \item   Wyznaczamy najniższy punkt $Q$ wyjściowego zbioru (jeżeli jest wiele o tej samej rzędnej -- bierzemy ten o najmniejszej odciętej).
        \item   Dodajemy $Q$ do zbioru punktów otoczki. 
        \item   Przeglądamy punkty zbioru w poszukiwaniu takiego, który wraz z ostatnim punktem otoczki tworzy najmniejszy kąt skierowany względem ostatniej znanej
                krawędzi otoczki. Dla pierwszego szukanego punktu, kąt namierzamy względem poziomu.
        \item   Znaleziony punkt dodajemy do zbioru punktów otoczki, jeżeli jest różny od $Q$.
        \item   Powtarzamy punkty 3 i 4 tak długo aż znalezionym punktem nie będzie $Q$.
        \item   Zwracamy listę punktów otoczki. 
    \end{enumerate}
    \subsubsection{Szczegóły}
    \begin{itemize}
        \item   Najniższy punkt wyjściowego zbioru (punkt 1) wyznaczamy w czasie liniowym, iterując po kolejnych punktach zbioru.
        \item   W celu wyznaczenia punktu wyspecyfikowanego w punkcie 3. nie obliczamy wartości odpowiedniego kąta. Zamiast tego, równoważnie, wyznaczamy punkt $P$, 
                który wraz z ostatnim znanym punktem otoczki $P_0$ tworzy wektor $\vec{P_0P}$ dla którego wszystkie pozostałe punkty zbioru są po lewej stronie. Robimy 
                to w czasie liniowym korzystając z znanych własności wyznacznika. 
    \end{itemize}
    \subsubsection{Złożoność}

    Zauważmy, że jeżeli otoczka jest $k$ - elementowa, to główna pętla algorytmu (punkty 3--4) wykonuje się $k$-razy. Każdy krok pętli (znalezienie odpowiedniego punktu $P$) zajmuje czas liniowy. 
    Pozostałe operacj w algorytmie zajmują co najwyżej czas liniowy. Zatem algrytm Jarvisa ma złożoność $O(nk)$.

    \subsubsection{Kod}

    \begin{lstlisting}[language=Python]
def jarvis(points: ListOfPoints) -> ListOfPoints:
EPS = 1e-8

convex_hull = []

start_idx = index_of_min(points, 1)

convex_hull.append(start_idx)

rand_idx = 0 if start_idx != 0 else 1

prev = start_idx

while True:
    imax = rand_idx
        
    for i in range(len(points)):
        if i != prev and i != imax:
            orient = orientation(
                        points[prev], 
                        points[imax], 
                        points[i], 
                        EPS
                     )
            if orient == -1:
                imax = i
                
            elif orient == 0 and \
                 (dist_sq(points[prev], points[imax]) < dist_sq(points[prev], points[i])):
                imax = i
                
    if imax == start_idx:
        break;

    convex_hull.append(imax)
    
    prev = imax

return points[convex_hull]
    \end{lstlisting}

    W ostatniej linii algorytmu, korzystamy z możliwości bibliteki \emph{numpy}.


\subsection{Algorytm górna-dolna}
    \subsubsection{Opis działania}
    \subsubsection{Szczegóły}
    \subsubsection{Złożoność}
    \subsubsection{Kod}



\subsection{Algorytm przyrostowy}
    \subsubsection{Opis działania}
    \subsubsection{Szczegóły}
    \subsubsection{Złożoność}
    \subsubsection{Kod}

\newpage

\subsection{Algorytm dziel i zwyciężaj}
    \subsubsection{Opis działania}
    \subsubsection{Szczegóły}
    \subsubsection{Złożoność}
    \subsubsection{Kod}



\subsection{Algorytm Chana}
    \subsubsection{Opis działania}
    cokolwiek \cite{markdeberg}
    \subsubsection{Szczegóły}
    \subsubsection{Złożoność}
    \subsubsection{Kod}
    
    
    
    \medskip

\printbibliography

\end{document}  