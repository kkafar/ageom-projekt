% K. Kafara
\documentclass[11pt]{article}
\usepackage[margin=1.35in]{geometry}
\usepackage{polski}
\usepackage[utf8]{inputenc}
\usepackage{amsmath}
\usepackage{amsthm}
\usepackage{graphicx}
\usepackage{pgfplots}
\usepackage{multirow}
\usepackage{pdfpages}
\usepackage{listings}
\usepackage[polish]{babel}
\usepackage[normalem]{ulem}
\usepackage[backend=biber,style=alphabetic,sorting=ynt]{biblatex}
\addbibresource{bibliography.bib}


\useunder{\uline}{\ul}{}
\pgfplotsset{compat=1.9}
\theoremstyle{remark} \newtheorem{definition}{def.}
\theoremstyle{definition} \newtheorem{twierdzenie}{tw.}
\newcommand{\bold}[1]{\textbf{#1}}
\newcommand{\bemph}[1]{\textbf{\emph{#1}}}
\newcommand{\eq}{\, = \,}
\newcommand{\apeq}{\, \approx \,}
\newcommand{\miunit}{\, \frac{v \cdot s}{a \cdot m}}
\newcommand{\abs}[1]{\left| #1 \right|}
\newcommand{\bunit}{\, \mu T}
\newcommand{\iunit}{\, mA}


\usepackage{xcolor}

\definecolor{codegreen}{rgb}{0,0.6,0}
\definecolor{codegray}{rgb}{0.5,0.5,0.5}
\definecolor{codepurple}{rgb}{0.58,0,0.82}
\definecolor{backcolour}{rgb}{0.95,0.95,0.92}

\lstdefinestyle{mystyle}{
    backgroundcolor=\color{backcolour},   
    commentstyle=\color{codegreen},
    keywordstyle=\color{magenta},
    numberstyle=\tiny\color{codegray},
    stringstyle=\color{codepurple},
    basicstyle=\ttfamily\footnotesize,
    breakatwhitespace=false,         
    breaklines=true,                 
    captionpos=b,                    
    keepspaces=true,                 
    numbers=left,                    
    numbersep=5pt,                  
    showspaces=false,                
    showstringspaces=false,
    showtabs=false,                  
    tabsize=2
}

\lstset{style=mystyle}

\author{K. Kafara\\Ł. Czarniecki}
\title{\textbf{Otoczka wypukła dla zbioru punktów w przestrzeni dwuwymiarowej}\\Dokumentacja projektu\\Algorytmy geometryczne}
\date{}

\begin{document}

\maketitle



\tableofcontents

\listoffigures

\listoftables

\newpage



\section{Informacje techniczne}

\subsection{Budowa programu}

Program złożony jest z następujących modułów: 

\begin{itemize}
    \item   \emph{lib} -- biblioteczny -- zawiera zbiór pomocniczych funkcji i struktur danych wykorzystywanych przez algorytmy.
    \item   \emph{pure} -- algorytmy w \emph{czystej postaci} tj. nie posiadające części wizualizacyjnej. 
    \item   \emph{vis} -- algorytmy wraz z kodem odpowiadającym za wizualizację
\end{itemize}


Poniżej przedstawiamy dokładny opis zawartości poszczególnych modułów. 

\subsubsection{Moduł \emph{lib}}

Moduł zawiera w sobie następujące podmoduły:

\begin{enumerate}
    \item   \emph{geometric\_tool\_lab.py} -- narzędzie graficzne dostarczone w ramach przedmiotu \emph{Algorytmy geometryczne}
    \item   \emph{getrand.py} -- zawiera funkcje generujące zbiory punktów różnych typów 
    \item   \emph{sorting.py} -- zawiera implementację iteracyjnej wersji algorytmu \emph{QuickSort} wykorzystywaną m.in w algorytmie Grahama
    \item   \emph{stack.py} -- zawiera klasę implementującą \emph{stos}
    \item   \emph{util.py} -- zawiera szereg funkcji pomocniczych wykorzystywanych przez zaimplementowane algorytmy
    \item   \emph{mytypes.py} -- zawiera definicje typów stworzone w celu zwiększenia czytelności kodu
\end{enumerate}


\subsubsection{Moduł \emph{pure}}

Moduł zawiera w sobie następujące podmoduły:

\begin{enumerate}
    \item   \emph{divide\_conq.py} -- implementacja algorytmu dziel i zwyciężaj
    \item   \emph{graham.py} -- implementacja algorytmu Grahama
    \item   \emph{increase.py} -- implementacja algorytmu przyrostowego
    \item   \emph{jarvis.py} -- implementacja algorytmu Jarvisa
    \item   \emph{lowerupper.py} -- implementacja algorytmu "górna-dolna"
\end{enumerate}

\subsubsection{Moduł \emph{vis}}

Moduł zawiera w sobie następujące podmoduły:

\begin{enumerate}
    \item   \emph{divide\_conq\_vis.py} -- implementacja algorytmu dziel i zwyciężaj wraz z kodem tworzącym wizualizację
    \item   \emph{graham\_vis.py} -- implementacja algorytmu Grahama wraz z kodem tworzącym wizualizację
    \item   \emph{increase\_vis.py} -- implementacja algorytmu przyrostowego wraz z kodem tworzącym wizualizację
    \item   \emph{jarvis\_vis.py} -- implementacja algorytmu Jarvisa wraz z kodem tworzącym wizualizację
    \item   \emph{lowerupper\_vis.py} -- implementacja algorytmu "górna-dolna" wraz z kodem tworzącym wizualizację
\end{enumerate}


\subsection{Wymagania techniczne}

\begin{enumerate}
    \item   Python 3.9.0 64-bit lub nowszy
    \item   Jupyter Notebook
\end{enumerate}

\subsection{Korzystanie z programu}

\subsubsection{Uruchomienie wizualizacji}

W celu uruchomienia wizualizacji algorytmów należy uruchomić notebook (poprzez Jupyter Notebook) \emph{program.ipynb},
a następnie zapoznać się z zamieszczoną tam instrukcją. 


\section{Oznaczenia i definicje}

Na potrzeby dalszych wywodów przyjmujemy w tym miejscu szereg oznaczeń i definicji:




\section{Problem}

Wyznaczyć otoczkę wypukłą podanego zbioru punktów płaszczyzny dwuwymiarowej. 

\section{Algorytmy}

\subsection{Algorytm Grahama}

    W celu opisania sposobu działania algorytmu Grahama, definiujemy następujacą relację $\preceq_Q$ określoną dla dowolnych dwóch punktów płaszczyzny $P_1$, $P_2$ względem 
    wybranego i ustalonego punktu odniesienia $Q$.

    \begin{eqnarray*}
        \label{eq:relacja-graham}
        P_1 \preceq_Q P_2 \, \Leftrightarrow \, 
        (\angle (P_1, Q, OX) < \angle (P_2, Q, OX)) 
        \lor
        (\angle (P_1, Q, OX) \eq \angle (P_2, Q, OX) \land d(P_1, Q) \leq d(P_2, Q)) 
    \end{eqnarray*}

    gdzie $d(P, Q)$ oznacza odległość od siebie dwóch dowolnych punktów płaszczyzny.

    Tak zdefiniowana relacja jest liniowym porządkiem (zwrotna, antysymetryczna, przechodnia i spójna).

    \subsubsection{Opis działania}

    \begin{enumerate}
        \item   Wyznaczamy najniższy punkt $Q$ wyjściowego zbioru (jeżeli jest wiele o tej samej rzędnej -- bierzemy ten o najmniejszej odciętej).
        \item   Ustawiamy go jako pierwszy element zbioru. 
        \item   Sortujemy pozostałe punkty względem relacji $\preceq_Q$.
        \item   Usuwamy wszystkie, poza najbardziej oddalonym od Q, punkty leżące na półprostej $QP$, dla każdego $P$
        \item   Kładziemy pierwsze 3 punkty zbioru na stos $S$. 
        \item   Iterujemy kolejno po punktach z posortowanego zbioru nie będących na stosie:
                Niech bieżącym punktem będzie P:

                \begin{enumerate}
                    \item   Dopóki $P$ nie jest po lewej stronie $S_{n-1}S_n$ wykonujemy (b)
                    \item   Uswamy punkt ze stosu. 
                    \item   Dodajemy $P$ na stos.
                \end{enumerate}
        \item Zwracamy zawartość stosu.
    \end{enumerate}


    \subsubsection{Szczegóły}

    \begin{itemize}
        \item   Najniższy punkt wyjściowego zbioru (punkt 1) wyznaczamy w czasie liniowym, iterując po kolejnych punktach zbioru. 
        \item   Wszystkie punkty leżacej na jednej prostej, poza najbardziej oddalonym od $Q$ usuwamy w czasie liniowym w następujący sposób:
                Iterując przez posortowaną tablicę, zaczynająć od indeksu $i := 1$, zapamiętujemy ostatni indeks na który wstawialiśmy $j$ (na początku $j := 1$).
                Jeżeli $Q$, $P_i$, $P_{i+1}$ są współliniowe to $i := i+1$. Jeżeli nie są współliniowe to $P_i$ wpisujemy na pozycję $j$, a następnie $j := j + 1$. Następnie, 
                w dalszej części algorytmu posługujemy się częścią tablicy $[0, \ldots, j - 1]$.
    \end{itemize}    


    \subsubsection{Złożoność}
    
    Operacją dominującą w algorytmie jest sortowanie -- realizowane w czasie $O(n \, lgn)$. Wybór punktu najniższego, redukcja punktów współlinowych oraz iterowanie (punkt 6, 
    zauważmy, że każdy punkt zbioru wyjściowego jest obsługiwany co najwyżej 2 razy -- gdy jest dodawany do otoczki i gdy jest ewentualnie usuwany) są realizowane w
    czasie $O(n)$. Algorytm Grahama ma zatem złożoność $O(n \, lgn)$.


    \subsubsection{Kod}


    \begin{lstlisting}[language=Python]
def get_point_cmp(ref_point: Point, eps: float = 1e-7) -> Callable:
    def point_cmp(point1, point2):
        orient = orientation(ref_point, point1, point2, eps)
        
        if orient == -1:
            return False
        elif orient == 1:
            return True
        elif dist_sq(ref_point, point1) <= dist_sq(ref_point, point2):
            return True
        else:
            return False

    return point_cmp


def graham(points: ListOfPoints) -> ListOfPoints:
    istart = index_of_min(points, 1)

    points[istart], points[0] = points[0], points[istart]    

    qsort_iterative(points, get_point_cmp(points[0]))

    i, new_size = 1, 1
    while i < len(points):
        while (i < len(points) - 1) \
        and \
        (orientation(points[0], points[i], points[i + 1], 1e-7) == 0):  
            i += 1
        
        points[new_size] = points[i]
        new_size += 1
        i += 1
    
    s = Stack()
    s.push(points[0])
    s.push(points[1])
    s.push(points[2])
    
    for i in range(3, new_size, 1):
        while orientation(s.sec(), s.top(), points[i], 1e-7) != 1:
            s.pop()

        s.push(points[i])

    return s.s[:s.itop+1]
    \end{lstlisting}



\subsection{Algorytm Jarvisa}
    \subsubsection{Opis działania}
    \begin{enumerate}
        \item   Wyznaczamy najniższy punkt $Q$ wyjściowego zbioru (jeżeli jest wiele o tej samej rzędnej -- bierzemy ten o najmniejszej odciętej).
        \item   Dodajemy $Q$ do zbioru punktów otoczki. 
        \item   Przeglądamy punkty zbioru w poszukiwaniu takiego, który wraz z ostatnim punktem otoczki tworzy najmniejszy kąt skierowany względem ostatniej znanej
                krawędzi otoczki. Dla pierwszego szukanego punktu, kąt namierzamy względem poziomu.
        \item   Znaleziony punkt dodajemy do zbioru punktów otoczki, jeżeli jest różny od $Q$.
        \item   Powtarzamy punkty 3 i 4 tak długo aż znalezionym punktem nie będzie $Q$.
        \item   Zwracamy listę punktów otoczki. 
    \end{enumerate}
    \subsubsection{Szczegóły}
    \begin{itemize}
        \item   Najniższy punkt wyjściowego zbioru (punkt 1) wyznaczamy w czasie liniowym, iterując po kolejnych punktach zbioru.
        \item   W celu wyznaczenia punktu wyspecyfikowanego w punkcie 3. nie obliczamy wartości odpowiedniego kąta. Zamiast tego, równoważnie, wyznaczamy punkt $P$, 
                który wraz z ostatnim znanym punktem otoczki $P_0$ tworzy wektor $\vec{P_0P}$ dla którego wszystkie pozostałe punkty zbioru są po lewej stronie. Robimy 
                to w czasie liniowym korzystając z znanych własności wyznacznika. 
    \end{itemize}
    \subsubsection{Złożoność}

    Zauważmy, że jeżeli otoczka jest $k$ - elementowa, to główna pętla algorytmu (punkty 3--4) wykonuje się $k$-razy. Każdy krok pętli (znalezienie odpowiedniego punktu $P$) zajmuje czas liniowy. 
    Pozostałe operacj w algorytmie zajmują co najwyżej czas liniowy. Zatem algrytm Jarvisa ma złożoność $O(nk)$.

    \subsubsection{Kod}

    \begin{lstlisting}[language=Python]
def jarvis(points: ListOfPoints) -> ListOfPoints:
EPS = 1e-8

convex_hull = []

start_idx = index_of_min(points, 1)

convex_hull.append(start_idx)

rand_idx = 0 if start_idx != 0 else 1

prev = start_idx

while True:
    imax = rand_idx
        
    for i in range(len(points)):
        if i != prev and i != imax:
            orient = orientation(
                        points[prev], 
                        points[imax], 
                        points[i], 
                        EPS
                     )
            if orient == -1:
                imax = i
                
            elif orient == 0 and \
                 (dist_sq(points[prev], points[imax]) < dist_sq(points[prev], points[i])):
                imax = i
                
    if imax == start_idx:
        break;

    convex_hull.append(imax)
    
    prev = imax

return points[convex_hull]
    \end{lstlisting}

    W ostatniej linii algorytmu, korzystamy z możliwości bibliteki \emph{numpy}.\\


\subsection{Algorytm górna-dolna}
    \subsubsection{Opis działania}
    \begin{enumerate}
        \item   Sortujemy punkty rosnąco po odciętych (w przypadku rówych, mniejszy jest punkt o mniejszej rzędnej).
        \item   Pierwsze dwa punkty z posortowanego zbioru wpisujemy do zbioru punktów otoczki górnej oraz dolnej.
        \item   Iterujemy po zbiorze punktów zaczynając od $i = 2$ (trzeciego punktu), niech $P$ będzie bieżącym punktem:
                \begin{enumerate}
                    \item   Dopóki górna (dolna) otoczka ma co najmniej 2 punkty i $P$ nie znajduje się po prawej (lewej) stronie odcinka skierowanego utworzonego przez ostatniej
                            dwa punkty otoczki (ostatni jest końcem odcinka), wykonujemy (b):
                    \item   Usuwamy ostatni punkt z otoczki górej (dolnej).
                    \item   Dodajemy $P$ do punktów otoczki górnej (dolnej).
                \end{enumerate}
        \item   Odwracamy kolejność wierzchołków w otoczce dolnej.
        \item   Łączymy zbioru punktów otoczki górnej oraz dolnej.
        \item   Zwracamy złączony zbiór punktów otoczki. 
    \end{enumerate}

    % \subsubsection{Szczegóły}

    \subsubsection{Złożoność}
    
    Dominującą operacją w algorytmie jest sortowanie realizowane w czasie $O(n \, lgn)$. Każdy krok pętli (dla wyznaczania otoczki górnej oraz dolnej) zajmuje czas stały. Zauważmy, że podobnie do 
    algorytmu Grahama każdy z punktów jest rozważany co najwyżej dwukrotnie -- w momencie dodania do otoczki i przy ewentualnym usunięciu ze zbioru punktów otoczki. Pozostałe operacje realizowane są w 
    czasie liniowym. Zatem algorytm "górna-dolna" ma złożoność $O(n \, lgn)$.

    \subsubsection{Kod}

\begin{lstlisting}[language=Python]
def lower_upper(point2_set: ListOfPoints) -> ListOfPoints:
if len(point2_set) < 3: return None

point2_set.sort(key = operator.itemgetter(0, 1))

upper_ch = [ point2_set[0], point2_set[1] ] 
lower_ch = [ point2_set[0], point2_set[1] ]

for i in range( 2, len(point2_set) ):
    while len(upper_ch) > 1 and orientation(upper_ch[-2], upper_ch[-1], point2_set[i]) != -1:
        upper_ch.pop()

    upper_ch.append(point2_set[i])

for i in range(2, len(point2_set) ):
    while len(lower_ch) > 1 and orientation(lower_ch[-2], lower_ch[-1], point2_set[i]) != 1:
        lower_ch.pop()

    lower_ch.append(point2_set[i])

lower_ch.reverse()
upper_ch.extend(lower_ch)

return upper_ch
\end{lstlisting}



\subsection{Algorytm przyrostowy}
    \subsubsection{Opis działania}
    Ogólne sformułowanie algorytmu ma postać:
    \begin{enumerate}
        \item   Dodajemy pierwsze 3 punkty do zbioru punktów otoczki. 
        \item   Iterujemy po pozostałych punktach. Niech $P$ będzie punktem bieżącym:
                \begin{enumerate}
                    \item   Jeżeli $P$ nie należy do wnętrza obecnie znanej otoczki wykonumejmy (b) oraz (c).
                    \item   Znajdujemy styczne do obecnie znanej otoczki poprowadzone przez punkt $P$. 
                    \item   Aktualizujemy otoczkę.
                \end{enumerate}
        \item   Zwracamy punkty otoczki. 
    \end{enumerate}


    Możemy go jednak sformułować inaczej, co pozwoli na uproszenie implementacji, przy zachowaniu takiego samego rzędu złożoności. 

    \begin{enumerate}
        \item   Sortujemy punkty rosnąco po odciętych (w przypadku rówych, mniejszy jest punkt o mniejszej rzędnej).
        \item   Dodajemy pierwsze 3 punkty do zbioru punktów otoczki, w takiej kolejności, aby były podane w kolejności odwrotnej 
                do ruchu wskazówek zegara. 
        \item   Iterujemy po pozostałych punktach. Niech $P$ będzie punktem bieżącym: 
                \begin{enumerate}
                    \item   Znajdujemy styczne do obecnie znanej otoczki poprowadzone przez punkt $P$.
                    \item   Aktualizujemy otoczkę. 
                \end{enumerate}
        \item   Zwracamy punkty otoczki.
    \end{enumerate}


    Dzięki wstępnemu posortowaniu puntków, omijamy konieczność testowania należenia $P$ do otoczki znanej w danym kroku algorytmu, ponieważ
    biorąc kolejny punkt mamy gwarancję, że nie należy on do wcześniej znanej otoczki. 



    \subsubsection{Szczegóły}
        \subsubsection*{Wyznaczanie stycznych}

        Styczne wyznaczamy w czasie logartymicznym względem liczby punktów należących do otoczki do której szukamy stycznych, 
        wykonując poszukiwanie binarne elementów skrajnych (najmniejszego i największego) względem następującego porządku, 
        określonego dla dowolnych dwóch punktów płaszczyzny $P_1$, $P_2$, względem ustalonego punktu $Q$: \\
        $P_1 >_Q P_2 \Leftrightarrow P_2$ znajduje się po lewej stronie odcinka skierowanego $QP_1$.

        W tak określonym porządku elementem największym będzie prawy punkt styczności $P_{max}$, ponieważ wszystkie inne punkty otoczki 
        znajdują się na lewo od prostej (stycznej) $QP_{max}$. Podobnie lewym punktem styczności będzie element najmniejszy 
        w zadanym porządku $P_{min}$. \\

        Wyznaczanie $P_{max}$ ($P_{min}$ wyznaczamy w sposób zupełnie analogiczny):\\
        Wprowadźmy najpierw potrzebne oznaczenia: \\
        $Q$ punkt zewnętrzny, przez który mają przechodzić styczne do otoczki.\\
        Niech otoczka będzie dana w postaci ciągu punktów $P_0, \ldots, P_{k-1}$ ($P_k := P_0$) będących współrzędnymi kolejnych wierzchołków w kolejności przeciwnej
        do ruchu wskazówek zegara.\\ 
        Krawędź skierowaną otoczki $e_i$ definiujemy następująco: $e_i := P_iP_{i+1}$. \\
        Krawędź $e_i := P_iP_{i+1}$ jest skierowana w górę $\Leftrightarrow$ $P_{i+1} >_Q P_i$.\\
        Krawędź $e_i$ jest skierowana w dół, jeżeli nie jest skierowana w górę. 

        \begin{enumerate}
            \item   Jeżeli $P_0$ jest większy od swoich obudwu sąsiadów, to go zwracamy jako element
                    największy. 
            \item   Definujemy indeksy $l := 0$, $r := k$
            \item   Dopóki nie znajdziemy elementu największego:
                    \begin{enumerate}
                    \item   Wyznaczamy indeks środkowego elementu $m := \left\lfloor \frac{l + k}{2} \right\rfloor$
                    \item   Jezeli $P_m$ jest elementem największym:
                            \begin{itemize}
                                \item   Zwracamy $P_m$.
                            \end{itemize}
                    \item   Jeżeli $e_l$ jest skierowana w górę:
                            \begin{enumerate}
                                \item   Jeżeli $e_m$ jest skierowana w dół 
                                        \begin{itemize}
                                            \item $r := m$
                                        \end{itemize}
                                \item   W przeciwnym przypadku
                                        \begin{enumerate}
                                            \item   Jeżeli $P_l >_Q P_m$:
                                                    \begin{itemize}
                                                        \item $r := m$
                                                    \end{itemize}
                                            \item   W przeciwnym przypadku
                                                    \begin{itemize}
                                                        \item $l := m$
                                                    \end{itemize}
                                        \end{enumerate}
                            \end{enumerate}
                    \item   W przeciwnym przypadku:
                            \begin{enumerate}
                                \item   Jeżeli $e_m$ jest skierowana w górę:
                                        \begin{itemize}
                                            \item $l := m$
                                        \end{itemize}
                                \item   W przeciwnym przypadku: 
                                        \begin{enumerate}
                                            \item   Jeżeli $P_m >_Q P_l$:
                                                    \begin{itemize}
                                                        \item $r := m$
                                                    \end{itemize}
                                            \item   W przeciwnym przypadku:
                                                    \begin{itemize}
                                                        \item $l := m$
                                                    \end{itemize}
                                        \end{enumerate}
                            \end{enumerate}
                    \end{enumerate}
        \end{enumerate}

        % Przeanalizujmy powyższy algorytm. 

        % Sprawdzamy najpierw czy $P_0$ nie jest szukanym punktem. Jest nim wtedy, gdy 
    \subsubsection{Złożoność}
        Posortowanie punktów zajmuje $O(n \, lgn)$. Każde wykonanie pętli zajmuje czas logarytmiczny względem liczebności otoczki
        znanej w danej iteracji. Zatem złożoność algorytmu jest rzędu $O(n \, lgn)$\\

    \subsubsection{Kod}

\begin{lstlisting}[language=Python]
def right_tangent(polygon: ListOfPoints, point: Point) -> Union[int, None]:
    n = len(polygon)

    if n < 3: return None

    if orientation(polygon[0], polygon[1], point) == 1 and orientation(polygon[-1], polygon[0], point) == -1:
        return 0
    
    left = 0
    right = n
    
    while True:
        mid = (left + right) // 2
        
        is_mid_down: bool = (orientation(polygon[mid], polygon[(mid + 1) % n], point) == 1)
        
        if is_mid_down and orientation(polygon[mid - 1], polygon[mid], point) == -1: 
            return mid % n
        
        is_left_up: bool = (orientation(polygon[left % n], polygon[(left + 1) % n], point) == -1)
        
        if is_left_up:
            if is_mid_down:
                right = mid
            else:
                if orientation(point, polygon[left], polygon[mid]) == 1:
                    right = mid
                else:
                    left = mid
        else:
            if not is_mid_down:
                left = mid
            else:
                if orientation(point, polygon[mid % n], polygon[left % n]) == 1:
                    right = mid
                else:
                    left = mid
                    
    
def left_tangent(polygon: ListOfPoints, point: Point) -> Union[Point, None]:
    n = len(polygon)
    
    if n < 3: return None

    left = 0
    right = n
    
    if orientation(point, polygon[1], polygon[0]) == 1 and orientation(point, polygon[-1], polygon[0]) == 1:
        return 0
    
    while True:
        mid = (left + right) // 2

        is_mid_down: bool = (orientation(point, polygon[mid % n], polygon[(mid + 1) % n]) == 1)
        
        if (not is_mid_down) and orientation(point, polygon[(mid - 1) % n], polygon[mid % n]) == 1:
            return mid % n
        
        is_left_down: bool = (orientation(point, polygon[left], polygon[(left + 1) % n]) == 1)
        
        if is_left_down:
            if not is_mid_down:
                right = mid
            else:
                if orientation(point, polygon[mid % n], polygon[left % n]) == 1:
                    right = mid
                else:
                    left = mid
        else:
            if is_mid_down:
                left = mid
            else:
                if orientation(point, polygon[left % n], polygon[mid % n]) == 1:
                    right = mid
                else:
                    left = mid

        
def increase_with_sorting(point2_set: ListOfPoints) -> Union[ListOfPoints, None]:
    if len( point2_set ) < 3: return None
    
    point2_set.sort(key = operator.itemgetter(0, 1))
    
    convex_hull = point2_set[:3]

    if orientation(convex_hull[0], convex_hull[1], convex_hull[2]) == -1:
        convex_hull[1], convex_hull[2] = convex_hull[2], convex_hull[1]
    
    for i in range(3, len( point2_set )):
        left_tangent_idx = left_tangent(convex_hull, point2_set[i])
        right_tangent_idx = right_tangent(convex_hull, point2_set[i])
        
        left_tangent_point = convex_hull[left_tangent_idx]
        right_tangent_point = convex_hull[right_tangent_idx]        

        deletion_side: Literal[-1, 1] = orientation(left_tangent_point, right_tangent_point, point2_set[i])

        if orientation(left_tangent_point, right_tangent_point, convex_hull[(left_tangent_idx + 1) % len(convex_hull)]) == deletion_side:
            step = 0
        else: 
            step = -1
            
        left = (left_tangent_idx + 1) % len(convex_hull)
        
        while convex_hull[left] != right_tangent_point:
            convex_hull.pop(left)
            left = (left + step) % len(convex_hull)
            
        convex_hull.insert(left, point2_set[i])

    return convex_hull
\end{lstlisting}

\medskip

\subsection{Algorytm dziel i zwyciężaj}
    \subsubsection{Opis działania}
    \begin{enumerate}
        \item   Sortujemy punkty rosnąco po odciętych (w przypadku rówych, mniejszy jest punkt o mniejszej rzędnej).
    \end{enumerate}
    \subsubsection{Szczegóły}
    \subsubsection{Złożoność}
    \subsubsection{Kod}



    \subsection{Algorytm dziel i zwyciężaj}
    \subsubsection{Opis działania}
    \subsubsection{Szczegóły}
    \subsubsection{Złożoność}
    \subsubsection{Kod}



\subsection{Algorytm Chana}
    \subsubsection{Opis działania}
    Główna część algorytmu Chana składa się  się z dwóch części:
    \begin{enumerate}
        \item   Pierwsza, która składa się  na :
                \begin{itemize}
                    \item   Podział zbioru punktów $Q$ na podzbiory $Q_i$ o w miarę równych
                            ilościach punktów w nich zawartych, z czego żaden nie 
                            zawiera więcej niż dane m. 
                    \item   Wyznaczenie otoczek $C_i$
                \end{itemize}
        \item   Druga polega na wykonaniu algorytmu na wzór Jarvisa, tylko na 
                otoczkach. Dokładniej mówiąc:
                \begin{itemize}
                    \item   Startujemy z najniższy wierzchołkeim z całego zbioru $Q$
                            i dodajemy go do finalnej otoczki jako pierwszy wierzchołek.
                    \item   Dla każdego punktu należącego do otoczki, możemy znaleźć
                            jego następnego sąsiada w otoczce idąc w kolejności przeciwnej
                            do ruchu wskazóek zegara. 
                            Aby to zrobić należy wybrać spośród zbioru punktów utworzonego z:
                            punktów tworzących prawą styczną z otoczkami $C_i$ dla rozważanego
                            wierzchołka, kolejnego punktu podotoczki do której dany punkt należy
                            taki wierzchołek, że wszystkie inne wierzchołki z tego zbioru są
                            na lewo od niego.
                    \item   W ten sposób wyznaczamy kolejne wierzchołki otoczki, dopóki
                            następnym wierzchołkiem otoczki nie jest jej pierwszy punkt.
                            Wtedy otoczka jest pełna i kończymy algorytm.
                \end{itemize}
    
        \item   Zwracamy punkty otoczki. 
    \end{enumerate}
    Jednakże nadżędną istotą powyższego algorytmu jest to, że wykona się on w drugiej
    części w co najwyżej m krokach(dany rozmiar podzbioru). Inaczej mówiąc m musi być
    większe bądź równe(w idealnym przypadku) liczbie punktów należących do otoczki k.
    Jeśli wykonamy m kroków i wciąż nie mamy otoczki, to przerywamy algorytm. I 
    próbujemy z większym m. Aby nie popsóć złożoności dużą ilością powtórzeń głównej
    części algorytmu najlepiej za każdym razem parametr m podnosić do kwadratu. W przypadku
    gdy m>=n po prustu za m przyjmujemy n. Wtedy algorytm chana sprowadza się do algorytmu 
    Grahama.
    \subsubsection{Szczegóły}  
    % \begin{itemize}
    %     \item   Najniższy punkt zbioru wyznaczamy w czasie liniowym. W trakcie podziału zpewniamy, że znajdzie się on w otoczce $C_0$ w indeksie $0$.
    %     \item   Wszystkie punkty leżacej na jednej prostej, poza najbardziej oddalonym od $Q$ usuwamy w czasie liniowym w następujący sposób:
    %             Iterując przez posortowaną tablicę, zaczynająć od indeksu $i := 1$, zapamiętujemy ostatni indeks na który wstawialiśmy $j$ (na początku $j := 1$).
    %             Jeżeli $Q$, $P_i$, $P_{i+1}$ są współliniowe to $i := i+1$. Jeżeli nie są współliniowe to $P_i$ wpisujemy na pozycję $j$, a następnie $j := j + 1$. Następnie, 
    %             w dalszej części algorytmu posługujemy się częścią tablicy $[0, \ldots, j - 1]$.
    % \end{itemize}     
    \subsubsection{Złożoność}
    Złożoność głównej części algorytmu.
    \begin{itemize}
        \item Złożoność pierwsszej części składa się na :
                \begin{itemize}
                    \item   Podział zbioru punktów na podzbiory O(n);
                    \item   Wyznaczenie otoczek dla podzbiorów. Mamy ceil(n/m) podzbiorów
                            rozmiaru m, dla każdego z nich wyznaczamy otoczkę algorytmem Grahama.
                            Algorytm Grahama działa O(nlog(n)). Więc łącznie mamy O(ceil(n/m)*mlog(m))=
                            O(nlog(m)).
                \end{itemize}
                Łącznie dla pierewszej części mamy O(nlog(m)), gdzie m jest wybranym maksymalnym
                 rozmiarem podzbiorów.

        \item Złożoność drugiej części składa się na :
                 \begin{itemize}
                     \item   Wyznaczenie następnego punktu dla każdego punktu z otoczki głównej o rozmiarze $k$
                     \item   Wyznacznie następnego punktu składa się na wyznaczenie dla każdej z m podotoczek
                            stycznej do tej podotoczki. Styczną wyznaczamy binary searchem w czasie $O(log(m))$
                            (otoczka $C_i$ ma co najwyżej $m$ wierzchołków). Otoczek jest $\left\lceil(n/m) \right\rceil$, a zatem czas
                            wyznaczenia kolejngo wierzchołka otoczki to $O(\lceil(n/m)\rceil log(m)$).    
                 \end{itemize}
                 Zakładając, że liczba wierzchołków otoczki $k \le m$(gdy $k > m$ przerywamy algorytm, więc złożoność pozostaje ta sama),
                  to ostatecznie mamy złożoność dla drugiej części rzędu : $O(k \left\lceil(n/m)\right\rceil log(m)) = O(nlog(m)) $w idealnym przypadku $O(nlog(k))$
    \end{itemize}

    Cała złożoność głównej części algorytmu, to $O(nlog(m))$ , gdzie $m$ jest wybraną wiellkości podzbioru.
    Złożonośc algorytmu dla próbowania algorytmu z kolejnymi $m$ postaci $2^{2^m}$ dla $m \ge 1$, to:
    Iteracja zako

    \subsubsection{Kod}
\begin{lstlisting}[language=Python]
from divide import *
from det import *
from tangentBothsides import *

def compr(p, q, current,
          accur=10 ** (-6)):  # jezeli p jest po prawej  odcinka [current,q] - jest 'wiekszy', to zwracamy 1
    if det(current, p, q) > accur:
        return -1
    elif det(current, p, q) < accur:
        return 1
    else:
        return 0

def nextvert(C, curr):  # dla danego punktu wspolzednymi z Q[i][j] jesli jest to punkt nalezacy do finalnej otoczki, to
    # zwraca nastepny punkt nalezacy do finalnej otoczki zadanego w takich samych wspolzednych Q[nxt[0]][nxt[1]]
    i, j = curr
    nxt = (i, (j + 1) % len(C[i]))
    for k in range(len(C)):
        t = tangent(C[i][j], C[k])
        if t != None and k != i and compr(C[nxt[0]][nxt[1]], C[k][t], C[i][j]) > 0 and (k, t) != (curr):
            nxt = (k, t)

    return nxt

def chanUtil(points, m):
    Q = divide(points, m)
    C = []
    for i in range(len(Q)):
        C.append(Graham(Q[i]))

    curr = (0, 0)
    ans = []
    i = 0
    while i < m:
        ans.append(C[curr[0]][curr[1]])
        if nextvert(C, curr) == (0, 0):
            return ans
        curr = nextvert(C, curr)
        i += 1

    return None

def chan(points):
    n = len(points)
    m = 4
    hoax = None
    while hoax == None:
        hoax = chanUtil(points, m)
        m = min(n, m * m)

    return hoax
\end{lstlisting}
    

\subsection{Algorytm QuickHull}
    \subsubsection{Opis działania}
    Algorytm QuickHull polega na rekurencyjnym wyznaczaniu kolejnych punktów otoczki. 
    \begin{enumerate}
        \item   Algorytm rozpoczynamy od wyznaczenie dwóch punktów skrajnych a,b - tj. o najmniejszej
                i największej współżędnej x-owej.
        \item   Następnie uruchamiamy funkcję rekurencyjnego znajdowania łuku należącego do otoczki
                między danym punktami należącymi do tej otoczki p,q  na prawo od odcinka |p,q|.
                Otoczką jest suma punktów a, wyniku działania funkcji rekurencyjnej dla odcinka |a,b|,b
                oraz wyniku działania funkcji rekurencyjnej dla |b,a|
        \item   Funkcja rekurencyjnego wyznaczenia łuku należącego do otoczki między punktami p i q polega na :
                \begin{itemize}
                    \item   Wyznaczeniu nabardziej oddalonego punktu na prawo od |p,q| jeśli są punkty po prawej.
                    \item   Jeśli nie ma takich punktów, to takiego łuku nie ma i zwracamy pustą tablicę. 
                    \item   W przeciwnym przypadku p,q należą do otoczki, to wyznaczony punkt skrajny r 
                            musi należeć do otoczki. 
                    \item   Skoro p,k,r należy do otoczki, to wszystkie wierzchołki wewnątrz trójkąta pkr napewno do najmniej
                            nie należą - usuwamy je.
                    \item   Szukany łuk, to suma działania tej samej funkcji dla punktów p,r, punktu r , oraz
                            wyniku tej funkcji dla punktów r,q w zadanej kolejności.
                    \item   Na koniec zwracamy wyznaczony w ten sposób łuk.
                \end{itemize}
    \end{enumerate}
    \subsubsection{Szczegóły}  
    \begin{itemize}
        \item   Rozpatrywane punkty p,q,r zawsze są podane w kolejności przeciwnej do ruchu wskazówek zegara.
                Aby usunąć punkty wewnątrz takich trójkątów należy dla każdego punktu z rozważanych sprawdzić,
                należy do danego trójkąta. 
        \item   Sprawdzenie, czy dany punkt należy do trójkąta pkr wykonujemy poprzez sprawdzenie, czy dla każdego
                z odcinków pr, rq, qp dany punkt znajduje się na lewo od tego odcinka, bądź jest z nim współliniowy.
        \item   Porównywanie odległości punktów r znajdujących się na prawo od odcinka pq wykonujemy za pomocą
                wyznacznika. Jest on wprostproporcjonlny do pola trójkąta rozpiętego na wektorach pq,pr. Ponieważ 
                odcinek pq ma stałą długośćdla każdego r, to wyznacznik ten jest wprostproporcjonalny do wysokości tego
                trójkąta opuszczonej na bok pq - odległości punktów.
    \end{itemize}     
    \subsubsection{Złożoność}
    Złożoność głównej części algorytmu.
    \begin{itemize}
        \item Złożoność pierwsszej części składa się na :
                \begin{itemize}
                    \item   Podział zbioru punktów na podzbiory O(n);
                    \item   Wyznaczenie otoczek dla podzbiorów. Mamy $\left\lceil(n/m)\right\rceil$ podzbiorów
                            rozmiaru $m$, dla każdego z nich wyznaczamy otoczkę algorytmem Grahama.
                            Algorytm Grahama działa $O(nlog(n))$. Więc łącznie mamy $O(\left\lceil(n/m)\right\rceil m log(m)) = O(nlog(m))$.
                \end{itemize}
                Łącznie dla pierewszej części mamy $O(nlog(m))$, gdzie m jest wybranym maksymalnym rozmiarem podzbiorów.

        \item Złożoność drugiej części składa się na :
                 \begin{itemize}
                     \item   Wyznaczenie następnego punktu dla każdego punktu z otoczki głównej o rozmiarze k;
                     \item   Wyznacznie następnego punktu składa się na wyznaczenie dla każdej z m podotoczek
                            stycznej do tej podotoczki. Styczną wyznaczamy z pomocą algorytmu wyszukiwania binarnego w czasie $O(log(m))$
                            (otoczka $C_i$ ma conajwyżej m wierzchołków). Otoczek jest $\left\lceil(n/m)\right\rceil$, a zatem czas
                            wyznaczenia kolejngo wierzchołka otoczki to $O(\left\lceil(n/m)\right\rceil log(m))$.    
                 \end{itemize}
                 Zakładając , że liczba wierzchołków otoczki k<=m(gdy k>m przerywamy algorytm, więc złożoność pozostaje ta sama),
                  to ostatecznie mamy złożoność dla drugiej części rzędu : O(k*ceil(n/m)*log(m)) = O(nlog(m)) w idealnym przypadku O(nlog(k))
    \end{itemize}

    Cała złożoność głównej części algorytmu, to $O(nlog(m))$ , gdzie m jest wybraną wiellkości podzbioru.
    Złożonośc algorytmu dla próbowania algorytmu z kolejnymi m postaci $2^{2^m}$ dla $m \ge 1$, to:
    Iteracja zako
    

    \subsubsection{Kod}
\begin{lstlisting}[language=Python]
def furthest(a, b, considering):
    n = len(considering)
    i = 0
    ans = None
    while i < n:
        if det(a, b, considering[i]) < 0:  # rozwazany wierzcholek jest po prawej stronie ab
            if ans == None or det(a, b, considering[i]) < det(a, b,
                                                              ans):  # |det(a,b,c)| = 1/2|ab|*h, gdzie h jest wysokoscia z c na ab
                ans = considering[i]
        i += 1
    return ans

def insideTriangle(a, b, c, i):
    if det(a, b, i) > 0 and det(b, c, i) > 0 and det(c, a, i) > 0:
        return True
    return False

def removeInner(a, b, c, considering):
    new=[]
    for i in considering:
        if not insideTriangle(a, b, c, i):
            new.append(i)
    considering.clear()
    considering+=new

def quickHullUtil(a, b, considering):
    if len(considering) == 0:
        return []

    c = furthest(a, b, considering)
    if c == None:
        return []
    considering.remove(c)

    removeInner(a, c, b, considering)
    return quickHullUtil(a, c, considering) +[c]+ quickHullUtil(c, b, considering)

def quickHull(points):
    a = min(points, key=lambda x: x[0])
    b = max(points, key=lambda x: x[0])

    considering = deepcopy(points)

    considering.remove(a)
    considering.remove(b)
\end{lstlisting}


    \medskip

% \printbibliographyy

\end{document}  